
%%
%% This is file `sample-sigconf.tex',
%% generated with the docstrip utility.
%%
%% The original source files were:
%%
%% samples.dtx  (with options: `sigconf')
%% 
%% IMPORTANT NOTICE:
%% 
%% For the copyright see the source file.
%% 
%% Any modified versions of this file must be renamed
%% with new filenames distinct from sample-sigconf.tex.
%% 
%% For distribution of the original source see the terms
%% for copying and modification in the file samples.dtx.
%% 
%% This generated file may be distributed as long as the
%% original source files, as listed above, are part of the
%% same distribution. (The sources need not necessarily be
%% in the same archive or directory.)
%%
%%
%% Commands for TeXCount
%TC:macro \cite [option:text,text]
%TC:macro \citep [option:text,text]
%TC:macro \citet [option:text,text]
%TC:envir table 0 1
%TC:envir table* 0 1
%TC:envir tabular [ignore] word
%TC:envir displaymath 0 word
%TC:envir math 0 word
%TC:envir comment 0 0
%%
%%
%% The first command in your LaTeX source must be the \documentclass
%% command.
%%
%% For submission and review of your manuscript please change the
%% command to \documentclass[manuscript, screen, review]{acmart}.
%%
%% When submitting camera ready or to TAPS, please change the command
%% to \documentclass[sigconf]{acmart} or whichever template is required
%% for your publication.
%%
%%
\documentclass[sigconf]{acmart}


%%
%% \BibTeX command to typeset BibTeX logo in the docs
\AtBeginDocument{%
  \providecommand\BibTeX{{%
    Bib\TeX}}}

%% Rights management information.  This information is sent to you
%% when you complete the rights form.  These commands have SAMPLE
%% values in them; it is your responsibility as an author to replace
%% the commands and values with those provided to you when you
%% complete the rights form.
\setcopyright{acmlicensed}
\copyrightyear{2024}
\acmYear{2024}
\acmDOI{XXXXXXX.XXXXXXX}

%% These commands are for a PROCEEDINGS abstract or paper.
\acmConference[Conference acronym DAC]{Make sure to enter the correct
  conference title from your rights confirmation emai}{June ,
  2024}{San Francisco, CA}
%%
%%  Uncomment \acmBooktitle if the title of the proceedings is different
%%  from ``Proceedings of ...''!
%%
%%\acmBooktitle{Woodstock '18: ACM Symposium on Neural Gaze Detection,
%%  June 03--05, 2018, Woodstock, NY}
\acmISBN{TBD} %978-1-4503-XXXX-X/18/06}


%%
%% Submission ID.
%% Use this when submitting an article to a sponsored event. You'll
%% receive a unique submission ID from the organizers
%% of the event, and this ID should be used as the parameter to this command.
%%\acmSubmissionID{123-A56-BU3}

%%
%% For managing citations, it is recommended to use bibliography
%% files in BibTeX format.
%%
%% You can then either use BibTeX with the ACM-Reference-Format style,
%% or BibLaTeX with the acmnumeric or acmauthoryear sytles, that include
%% support for advanced citation of software artefact from the
%% biblatex-software package, also separately available on CTAN.
%%
%% Look at the sample-*-biblatex.tex files for templates showcasing
%% the biblatex styles.
%%

%%
%% The majority of ACM publications use numbered citations and
%% references.  The command \citestyle{authoryear} switches to the
%% "author year" style.
%%
%% If you are preparing content for an event
%% sponsored by ACM SIGGRAPH, you must use the "author year" style of
%% citations and references.
%% Uncommenting
%% the next command will enable that style.
%%\citestyle{acmauthoryear}

%\documentclass[11pt]{article} 
%\usepackage{times}

%\usepackage{pifont}
%\usepackage{bm}
%\usepackage{epsfig}
%\usepackage{psfig}
%\usepackage{tabularx}
%\usepackage{amssymb}
%\usepackage{latexsym}
%\usepackage{amsmath}
%\usepackage{delarray}
%\usepackage{moreverb}
%\usepackage{xspace}
%\usepackage{algorithm}
%\usepackage{algorithmic}

%\usepackage{lscape}
%\usepackage{changebar}
%\usepackage{graphicx}
%\usepackage{graphics}
%\usepackage{mflogo}
%\usepackage{xspace}
%\usepackage{texnames}
%\usepackage{rotating}
%\usepackage{alltt}

%\usepackage{amsmath}
%\usepackage{epsfig}
%\usepackage{booktabs,paralist}

\def\firmopsa{d}
\def\firmopsb{d}
\newcommand\AssignTags{TagIt}


% Definitions
% -----------
\def\x{{\mathbf x}}
\def\L{{\cal L}}
\def\spiral{SPIRAL\xspace}

\newcommand{\pred}{{}}

\newcommand{\tensor}[0]{\otimes}
\newcommand{\inv}[1]{{1{/}{#1}}}
%\newcommand{\inv}[1]{$\textstyle\frac{1}{#1}$}
\newcommand{\DFT}{\operatorname{\bf DFT}}
\newcommand{\WHT}{\operatorname{\bf WHT}}
\newcommand{\FIR}{\operatorname{\bf FIR}}
\newcommand{\one}[0]{{I}}
\newcommand{\eq}[0]{{=}}
\newcommand{\Tensor}[2]{#1{\tensor}#2}
\newcommand{\TensorT}[3]{{#1{\tensor}#2}_{#3}}
\newcommand{\SPw}[2]{{\prec}#1,#2{\succ}}
\newcommand{\SPr}[2]{{<}#1,#2{>}}

\newcommand{\WTensorI}[2]{\Tensor{\WHT_{#1}}{\one_{#2}}}
\newcommand{\ITensorW}[2]{\Tensor{\one_{#1}}{\WHT_{#2}}}
\newcommand{\Wht}[2]{(\WTensorI{\WHT_{#1}}{#2})(\ITensorW{#1}{#2})}

\newcommand{\Times}[2]{{#1}{\times}{#2}}
\newcommand{\EQ}{{=}}
\newcommand{\Space}{{ S}}
\newcommand{\PEQ}{{{+}{=}}}
\newcommand{\SEQ}{{{-}{=}}}
\newcommand{\h}{\psi}
\newcommand{\balpha}{\bm{\alpha}}
\newcommand{\bgamma}{\bm{\gamma}}
\newcommand{\bbeta}{\bm{\beta}}
\newcommand{\bmu}{\bm{\mu}}
\newcommand{\bsigma}{\bm{\sigma}}


\newcommand{\f}{\varphi}
\newcommand{\Z}{\mathbb{Z}}
\newcommand{\A}{\mathbb{A}}
\newcommand{\B}{{\bf B}}
\newcommand{\X}{\mathbb{X}}
\newcommand{\T}{\mathbb{T}}
%\newcommand{\Q}{\mathbb{Q}}
\newcommand{\N}{\mathbb{N}}
\newcommand{\R}{\mathbb{R}}
\newcommand{\CC}{{\Gamma}}
\newcommand{\D}{\mathbb{D}}
\newcommand{\M}{\mathbb{M}}
\newcommand{\F}{\mathbb{F}}
\newcommand{\ST}{\mathbb{S}}
\newcommand{\Vc}[1]{{\boldsymbol #1}}
\newcommand{\fl}[1]{{\lfloor {#1} \rfloor}}
\newcommand{\cl}[1]{{\lceil {#1} \rceil}}
\newcommand{\half}[1]{{\frac{#1}{2}}}
\newcommand{\q}[1]{\Times{\cl{#1}}{\cl{#1}}}
\newcommand{\qq}[1]{\Times{\cl{#1}}{\fl{#1}}}
\newcommand{\qqq}[1]{\Times{\fl{#1}}{\cl{#1}}}
\newcommand{\qqqq}[1]{\Times{\fl{#1}}{\fl{#1}}}
\newcommand{\size}[1]{\sigma(#1)}
\newcommand{\Size}[3]{\sigma(\Vc{#1})\EQ\Times{#2}{#3}}
\newcommand{\n}{(\frac{n}{2})}
\newcommand{\m}{(\frac{n-1}{2})}
\newcommand{\Ceil}[1]{{\cl{\half{#1}}}}
\newcommand{\Floor}[1]{{\fl{\half{#1}}}}

\newcommand{\Endash}{{--}}
\newcommand{\Emdashbegin}{{---}}
\newcommand{\Emdashend}{{---}~}

\newcommand{\Transducer}[1] {{\mathbf #1 }}
\newcommand{\Triangle}[1] {{\vartriangle\!\! #1 }}


\newcommand{\AO}  {\Times{\Ceil{m}}{\Ceil{n}}}
\newcommand{\AI} {\Times{\Ceil{m}}{\Floor{n}}}
\newcommand{\AII}{\Times{\Floor{m}}{\Ceil{n}}}
\newcommand{\AIII} {\Times{\Floor{m}}{\Floor{n}}}
\newcommand{\BO}  {\Times{\Ceil{n}}{\Ceil{p}}}
\newcommand{\BI} {\Times{\Ceil{n}}{\Floor{p}}}
\newcommand{\BII}{\Times{\Floor{n}}{\Ceil{p}}}
\newcommand{\BIII} {\Times{\Floor{n}}{\Floor{p}}}
\newcommand{\CO}  {\Times{\Ceil{m}}{\Ceil{p}}}
\newcommand{\CI} {\Times{\Ceil{m}}{\Floor{p}}}
\newcommand{\CII}{\Times{\Floor{m}}{\Ceil{p}}}
\newcommand{\CIII} {\Times{\Floor{m}}{\Floor{p}}}

\newcommand{\I}[2]{\Times{\Ceil{#1}}{\Ceil{#2}}}
\newcommand{\II}[2]{\Times{\Ceil{#1}}{\Floor{#2}}}
\newcommand{\III}[2]{\Times{\Floor{#1}}{\Ceil{#2}}}
\newcommand{\IV}[2]{\Times{\Floor{#1}}{\Floor{#2}}}


\newcommand{\Q}[2]{\Vc{#1}_{#2}}

\newcommand{\QAO}{{\Vc{A}_{0}}}
\newcommand{\QAI}{\Vc{A}_{1}}
\newcommand{\QAII}{\Vc{A}_{2}}
\newcommand{\QAIII} {\Vc{A}_{3}}

\newcommand{\QBO}  {\Vc{B}_{0}}
\newcommand{\QBI} {\Vc{B}_{1}}
\newcommand{\QBII}{\Vc{B}_{2}}
\newcommand{\QBIII} {\Vc{B}_{3}}

\newcommand{\QCO}  {\Vc{C}_{0}}
\newcommand{\QCI} {\Vc{C}_{1}}
\newcommand{\QCII}{\Vc{C}_{2}}
\newcommand{\QCIII} {\Vc{C}_{3}}
\newcommand{\FIGsize}[2]{{width={#1}, height={#2}}}

\newcommand{\Mod}[2]{{#1}{\text{ \bf mod }}{#2}}

\newcommand{\TableRef}[1]{Table \ref{#1}}
\newcommand{\FigureRef}[1]{Figure \ref{#1}}
\newcommand{\SectionRef}[1]{Section \ref{#1}}
\newcommand{\EquationRef}[1]{Equation \ref{#1}}
\newcommand{\OnPageRef}[1]{on page \pageref{#1}}

% Formatting
% ----------
%\setlength{\evensidemargin}{0mm}
%\setlength{\oddsidemargin}{0mm}
%\setlength{\textwidth}{6.5in}
%\setlength{\textheight}{9.5in}
%\setlength{\topmargin}{0in}
%\setlength{\headheight}{0in}
%\setlength{\headsep}{0mm}

\newenvironment{noinds_itemize}{\begin{list}{$\bullet$}
{\setlength{\rightmargin}{0em}
\setlength{\leftmargin}{1.2em}
\setlength{\itemsep}{0em}
\setlength{\topsep}{0em}
\setlength{\parsep}{0em}}}{\end{list}}



\newcommand{\mypar}[1]{{\bf #1.}}
\newcommand{\mytag}[2]{[#1]_{#2}}
\sloppy

%% jan 01, 2009 - dasdan - removed 'final' pdflatex cannot work with it.
%\usepackage[final=true,bookmarks=true,bookmarkstype=toc,colorlinks=true, linkcolor=blue,citecolor=blue]{hyperref}
%\usepackage[bookmarks=true,bookmarkstype=toc,colorlinks=true, linkcolor=blue,citecolor=blue]{hyperref}

% width fraction, mflops, relative, caption, label
\newcommand{\doublefigure}[5]{{\begin{figure}%
\centering %
\includegraphics[width=#1\linewidth]{#2}
\includegraphics[width=#1\linewidth]{#3}%\figurebox{#1\linewidth}{}{}[#2]
%\figurebox[#1\linewidth]{}{}[#3] %
\caption{#4}%bio                 
\label{#5}%
\end{figure}}}

\newcommand{\Doublefigure}[5]{{\begin{figure*} %
\centering %
%\includegraphics[height=#1\linewidth,angle=-90]{#2}%
%\includegraphics[height=#1\linewidth,angle=-90]{#3} %
\includegraphics[width=#1\linewidth]{#2} \\  %
\includegraphics[width=#1\linewidth]{#3}%
%\figurebox{#1\linewidth}{}{}[#2]
%\figurebox{#1\linewidth}[#3] %
\caption{#4}%
\label{#5}%
\end{figure*}}}

\newcommand{\DDoublefigure}[5]{{\begin{figure*} %
\centering %
%\includegraphics[height=#1\linewidth,angle=-90]{#2}%
%\includegraphics[height=#1\linewidth,angle=-90]{#3} %
\includegraphics[width=#1\linewidth, angle=90]{#2}   %
\includegraphics[width=#1\linewidth, angle=90]{#3}%
%\figurebox{#1\linewidth}{}{}[#2]
%\figurebox{#1\linewidth}[#3] %
\caption{#4}%
\label{#5}%
\end{figure*}}}

\newcommand{\DDDoublefigure}[5]{{\begin{figure}[htb] %
\centering %
%\includegraphics[height=#1\linewidth,angle=-90]{#2}%
%\includegraphics[height=#1\linewidth,angle=-90]{#3} %
\includegraphics[width=#1\linewidth, angle=90]{#2}   %
\includegraphics[width=#1\linewidth, angle=90]{#3}%
%\figurebox{#1\linewidth}{}{}[#2]
%\figurebox{#1\linewidth}[#3] %
\caption{#4}%
\label{#5}%
\end{figure}}}

\newcommand{\Singlefigure}[4]{{\begin{figure*} %
\centering %
\includegraphics[width=#1\linewidth]{#2}%\figurebox{#1\linewidth}{}{}[#2]
\caption{#3}%
\label{#4}%
\end{figure*}}}

\newcommand{\SSinglefigure}[4]{{\begin{figure} %
\centering %
\includegraphics[width=#1\linewidth,angle=90]{#2}%\figurebox{#1\linewidth}{}{}[#2]
\caption{#3}%
\label{#4}%
\end{figure}}}

\newcommand{\singlefigure}[4]{{\begin{figure}[htb] %
\centering %
\includegraphics[width=#1\linewidth]{#2}%
%\includegraphics[height=#1\linewidth,angle=-90]{#3} %
%\figurebox{#1\linewidth}{}{}[#2]\\
\caption{#3}%
\label{#4}%
\end{figure}}}

\newcommand{\singlefigurerotate}[4]{{\begin{figure}[htb] %
\centering %
\includegraphics[width=#1\linewidth,angle=-90]{#2}%
%\includegraphics[height=#1\linewidth,angle=-90]{#3} %
%\figurebox{#1\linewidth}{}{}[#2]\\
\caption{#3}%
\label{#4}%
\end{figure}}}

\newcommand{\orthogonal}{%
\mathrel{\raisebox{.1em}{%     
\reflectbox{\rotatebox[origin=c]{90}{$\models$}}}}}


%%
%% end of the preamble, start of the body of the document source.
\begin{document}

%%
%% The "title" command has an optional parameter,
%% allowing the author to define a "short title" to be used in page headers.
\title{Weight Block Sparsity: Training, Compilers, and Accelerators}

%%
%% The "author" command and its associated commands are used to define
%% the authors and their affiliations.
%% Of note is the shared affiliation of the first two authors, and the
%% "authornote" and "authornotemark" commands
%% used to denote shared contribution to the research.
\author{Anonymous} 
%\author{Paolo D'\!Alberto}
%\author{Taehee Jeong}
%\author{Akshay Jain}
%\author{Shreyas Manjunath}
%\author{Mrinal Sarmah}
%\author{Samuel Hsu}
%\author{Nitesh Pipralia}

%%
%% By default, the full list of authors will be used in the page
%% headers. Often, this list is too long, and will overlap
%% other information printed in the page headers. This command allows
%% the author to define a more concise list
%% of authors' names for this purpose.
%\renewcommand{\shortauthors}{ D'\!Alberto et al.}
\renewcommand{\shortauthors}{ Anonymous  et al.}

%%
%% The abstract is a short summary of the work to be presented in the
%% article.
\begin{abstract}
We present the main ideas about a vertical system where convolution
and matrix multiplication weights can be trained to exploit an 8x8
block sparsity, compilers recognize such a sparsity for both data
compaction and computation splitting into threads. If we take a
Resnet50, we can reduce the weight by half with little accuracy
loss. We can achieve speeds similar to an hypothetical Resnet25. We
shall present performance estimates by accurate and complete code
generation for a small and efficient set of AIE2 (AMD Versal FPGAs)
using Resnet50, Inception V3, and VGG16.
\end{abstract}

%%
%% The code below is generated by the tool at http://dl.acm.org/ccs.cfm.
%% Please copy and paste the code instead of the example below.
%%

%%
%% Keywords. The author(s) should pick words that accurately describe
%% the work being presented. Separate the keywords with commas.
\keywords{AI, FPGA, Performance, Sparsity, and Tools}
%% A "teaser" image appears between the author and affiliation
%% information and the body of the document, and typically spans the
%% page.

\received{TBD}
\received[revised]{TBD}
\received[accepted]{TBD}

%%
%% This command processes the author and affiliation and title
%% information and builds the first part of the formatted document.
\maketitle


\section{Introduction}
\label{sec:introduction}

We explain what we mean for block sparsity and for a vertical
solution. Block sparsity is an intuitive concept but it is also a
little misunderstood. Take a matrix multiplication in Equation
\ref{eq:mat}
\begin{equation}
  \label{eq:mat}
  \begin{pmatrix}
    \Vc{C}_0 & \Vc{C}_1 \\
    \Vc{C}_2 & \Vc{C}_3 \\ 
  \end{pmatrix} = 
  \begin{pmatrix}
    \Vc{A}_0 & \Vc{A}_1 \\
    \Vc{A}_2 & \Vc{A}_3 \\ 
  \end{pmatrix}\\
  \begin{pmatrix}
    \Vc{0}   & \Vc{B}_1 \\
    \Vc{B}_2 & \Vc{0} \\ 
  \end{pmatrix}\\
\end{equation}
This is the computation {\small \begin{equation} \Vc{C}_0 = \Vc{A}_{1}
    \Vc{B}_{2}; \; \Vc{C}_1 = \Vc{A}_{0} \Vc{B}_{1}; \; \Vc{C}_2 =
    \Vc{A}_{3} \Vc{B}_{2}; \; \Vc{C}_3 = \Vc{A}_{2} \Vc{B}_{1}
\end{equation}}
and in general with proper $\gamma_i$ (i.e., a mask)
\begin{equation}
  \Vc{C}_i = \sum_{k=0}^1 \Vc{A}_{2i+ k} \big(\gamma_{2*k+i} \Vc{B}_{2*k+i}\big)
\end{equation}
Where the matrix $\Vc{B}$ is constant, diagonal, and each submatrix
$\Vc{B_2}$ and $\Vc{B}_1$ can split further down and may have even
smaller zero blocks. In this work, we chose the basic block of
$\Vc{B}_i = 8\times 8$. It is a great starting point for architectures
based on AMD AIE2 products and we support others.  For example,
\begin{equation}
  \Vc{B} = \dot{\sum}_i \gamma_i \Vc{B}_i, \;\; \gamma_i \in \{0,1\} 
\end{equation}

This is a well known data structure in the sparse computation field.
We can use {\em compress block row} (CBR) or {\em }column format
(CBC). There are standard matrix sparse-matrix multiplication
interfaces and algorithms for CPU and GPUs using this data format
(where only one operand is sparse or both) \cite{rocSPARSE,cuSPARSE}.

We explore training techniques (PyTorch and the Keras).  The most
successful is the simplest. We take a pre-trained model. We compute a
$\Gamma$ per layer using a function to determine the more likely zeros
blocks (using a Norm). Then we train the model till convergence or
accuracy achieved. We take the sparse model and we quantize to 8-bit
integer computations with the Vitis-AI quantizer. The final model is a
XIR quantized model (Xilin intermediate representation). See Section
\ref{sec:training}. We have a custom compiler that takes the XIR model
and an abstraction of a connected set of AIE2. See Section
\ref{sec:compiler}. The compiler computes the maximum sub-volume
computation per core. By heuristics and following a schedule, it
computes a memory allocation in memtile for input, outputs, and
weights. It formats the weights exploiting spatial distribution to
memtiles and cores. We generate all the explicit communications
between DDR, memtile, and cores. We Know the subproblem sizes per
core, the computation throughput and with a clear specification of
what is executed in parallel. Then, we can estimate the execution time
per layer and entire network with an accuracy closer to a simulation.
We compute time estimates for all parts of the computation. We will
show estimates for three CNN models and eight different AIE designs;
see Section \ref{sec:experiments}.

In the following Section \ref{sec:motivation}, we start with a
quantitative measure about the advantages of block sparsity.

\section{Block-Sparse Matrix-Matrix Multiplication}
\label{sec:motivation}

As a thought experiment, consider $\Gamma$ and $\Omega$ two matrices
in $\{0,1\}^{N\times N}$.
\begin{equation}
  \Vc{C} = (\Gamma \Vc{A}) * (\Omega \Vc{B})^t
\end{equation}
More precisely, consider non-zero blocks of size $k\times k$ so that
\begin{equation}
  \Vc{C}_{i*N+j} = \sum_k ( \gamma_{i*N+k} \Vc{A}_{i*N+k} ) (\dot{\omega}_{j*N+k} \dot{\Vc{B}}_{j*N+k})
\end{equation}


Thanks to the sparsity and if we store only non-zeros, then
$\gamma_{i*N+k}$ and $\dot{\omega}_{j*N+k}$ are at contiguous. There
will be a meaningful product to compute if and only if $\gamma_{i*N+k}
=1$ and $\dot{\omega}_{j*N+k} =1$.  We merge-sort these vectors.  See
how the sparse sparse matrix multiplication using {\em Coordinate
  Block Structure} (COO) is applied in Figure \ref{fig:block}.
\begin{comment}
  We provide software to reproduce this. % \cite{PaoloG2020}.
\end{comment}
Now, assume we want to achieve a fixed sparsity (i.e., density) of
50\% for a square matrix of size $N$ and we choose the block size $k
\times k$. The larger $k$ is, the smaller the overhead will be.  The
relative performance of the $k^3$ multiplication is better as $k$ get
larger because spatial, temporal locality, and further optimized code
for a constant/parameterized such as $k$.

\doublefigure{0.80}{1x1.png}{8x8.png}{Block 1x1 and 8x8
  performance}{fig:block}

In Figure \ref{fig:block}, we present two scatter plots: on the
abscissa the effective multiplication-and-addition number, on the
ordinate the performance in GFLOPS, when the sparse matrix with dense
block is $1\times 1$ (above) and $8\times8$ (below). Given the same
problem, we deploy more threads and thus the Jenga effect.  With the
same number of effective operations, the block permits and exploits
higher GFLOPS per effective operation (Float is 2x faster than Double
precision and this can be emphasized further
\cite{Gray2017GPUKF,li2023popsparse} and \cite{pmlr-v119-kurtz20a}).


\section{Block Sparsity: Training and Quantization}
\label{sec:training}

In Convolutional Neural Networks, the two main operations are
convolutions/correlations and fully connected layers (matrix
multiplication). The block sparsity we plan to deploy is not naturally
recurring.  We must train the network for it.

A convolution has a weight tensor in four dimension: $\Vc{W} \in
\R^{c_{out}\times h \times k \times c_{in}}$. In the hyperplane of the
$h$ and $k$, we can simplify the weight as $\dot{\Vc{W}} \in
\R^{c_{out} \times c_{in}}$ and block sparsity can be simply described
by a mask $\Gamma\dot{\Vc{W}}$. Although, we speak of a $8\times 8$ of
non-zeros, this is in practice a $8\times h\times k\times 8$
block. For the matrix multiply $h=k=1$, there is no difference from
the previous discussions. We explain the training process.


\subsection{Keras}
We shall provide a repository using Keras \cite{chollet2015keras}
where we implements the contents of this section. %\cite{PaoloK2020}.

We target convolutions only and without quantization. The idea is
simple: we take any model and we create a copy where we enhance the
convolution with a (non-trainable) $\Gamma$. A convolution will have
three parameters (saving the model into a different format).  The
forward computation is modified so that the weights used for
convolution are $\Gamma\Vc{W}$. We assume the backward computation
(i.e., gradient) is done automatically from the forward
definition. There is no need to change the bias. For example, we take
Resnet50 from the Keras application repository, we start with a
$\Gamma=1$, and we trained one epoch using imagenet repository
\cite{deng2009imagenet}.  The goal is to choose $\Gamma$ in such a way
we achieve the required sparsity and the minimum loss in accuracy. We
can tested different approaches such as incremental, Fisher measure,
Hessian, diagonal Hessian, and custom penalty losses. We will give
full description in another venue.

\subsection{$\Gamma$ Chosen Once and Full Training Ahead: PyTorch}
\label{sec:one-mask}
\label{sec:pytorch}
Take a convolution with $\Gamma = 1$ and weights $\Vc{W}$. For each
$\gamma_i$, this will be representative of a block $\Vc{W}_i \in \R^{8
  \times h \times w \times 8} \sim \R^{8\times 8}$. We can choose the
$\Vc{W}_i$ using a measure of importance:
\begin{itemize}
  \item $L_2 = \sqrt{\sum_k w_k^2}$ with $w_k \in \Vc{W}_i$,
  \item $L_1 = \sum_k |w_k|$ as above,
  \item Variance $\sigma^2 = \frac{1}{64}\sum_k (w_k -\mu)^2$ with
    $\mu = \frac{1}{64}\sum w_k, w_k \in \Vc{W}_i $ or $\frac{1}{N}\sum
    w_k, w_k \in \Vc{W}$. In signal processing $\sigma^2$ is the power
    of the signal.
\end{itemize}
We can then sort them in ascending order. We set the first half to
zero.  Then we start re-training. We do this for the entire network or
for one convolution at a time.

In Table \ref{tab_acc}, we show the results by using one-time mask
and full training: VGG-16, ResNet-50, Inceptionv3 on ImageNet20 (20
classes) and ImageNet1k (1000 classes).  We use three samples per
class for the validation accuracy for ImageNet1k data set; instead, we
use 50 samples per class for ImageNet20. Fine-tuning sparse networks
on the original ImageNet data-set \cite{deng2009imagenet} is
expensive. To reduce the training time, we chose 20 classes (from the
original 1000 classes) with the least number of images per class in
the training data-set and this choice will affect the accuracy because
there are fewer samples for re-training.


\begin{table}[ht]
\caption{Accuracies of the sparsity models}
\label{tab_acc}
\begin{center} 
\scalebox{0.9}
{
\begin{tabular}{|l|c|c|c|c|c|}
\hline
\rule[-1ex]{0pt}{3.5ex}  Model & Dataset & Baseline  & \multicolumn{3}{c|}{Sparsity}\\
\rule[-1ex]{0pt}{3.5ex}  {} & {} & Acc.(\%) & block & ratio (\%) & Acc.(\%)    \\\hline\hline
\rule[-1ex]{0pt}{3.5ex}  Inception-v3 & ImageNet1k & 77.2 & 8x8 & 50 & 75.5  \\\hline
\rule[-1ex]{0pt}{3.5ex}  ResNet-50 & ImageNet1k & 76.7 & 8x8 & 50 & 74.6  \\\hline
\rule[-1ex]{0pt}{3.5ex}  VGG-16    & ImageNet1k & 70.6 & 8x8 & 50 & 69.7  \\\hline \hline
\rule[-1ex]{0pt}{3.5ex}  ResNet-50 & ImageNet20 & 96.1 & 8x8 & 25 & 95.1  \\\hline
\rule[-1ex]{0pt}{3.5ex}  ResNet-50 & ImageNet20 & 96.1 & 8x8 & 50 & 92.0  \\\hline
\rule[-1ex]{0pt}{3.5ex}  ResNet-50 & ImageNet20 & 96.1 & 8x8 & 75 & 87.1  \\\hline
\rule[-1ex]{0pt}{3.5ex}  ResNet-50 & ImageNet20 & 96.1 & 1x1 & 25 & 96.0  \\\hline
\rule[-1ex]{0pt}{3.5ex}  ResNet-50 & ImageNet20 & 96.1 & 1x1 & 50 & 95.6  \\\hline
\rule[-1ex]{0pt}{3.5ex}  ResNet-50 & ImageNet20 & 96.1 & 1x1 & 75 & 93.5  \\\hline
\rule[-1ex]{0pt}{3.5ex}  VGG-16    & ImageNet20 & 92.0 & 8x8 & 50 & 89.6  \\\hline
\rule[-1ex]{0pt}{3.5ex}  VGG-16    & ImageNet20 & 92.0 & 1x1 & 50 & 92.3  \\\hline
\rule[-1ex]{0pt}{3.5ex}  VGG-16    & ImageNet20 & 92.0 & 1x1 & 75 & 91.7  \\\hline
\end{tabular}\vspace{-20pt}
}
\end{center}
\end{table}

Classification accuracy on ImageNet1k drops by only 1 - 2\% after
applying 50\% sparsity with a $8\times 8$ block (this is without any
quantization). We experiment with different block shapes such as
$16\times 4$ and $4\times 16$ on ResNet-50, but the accuracy is
slightly worse. Fine-grained sparsity ($1\times 1$ block or
unstructured) does not sacrifice any accuracy (i.e., almost any).  We
use the sparsified models, we quantize them using Vitis AI, and we use
them for time estimates (i.e., Section \ref{sec:experiments}).


\section{The Compiler and its Code generation for AIE}
\label{sec:compiler}
We can take a PyTorch/Keras model, quantize it using Vitis AI, and
create an intermediate representation that we call Xilinx Intermediate
Representation (XIR). XIR is a graph where each node is an operation
that reads tensors and writes one tensor.  A convolution has one
quantized input.  We use the tensor layout format BHWC, the tensors
are represented in INT8 with a position where the fraction starts
(power of two scale). It computes a tensor using the same layout and
with a proper scale. The weights and bias are properties of the
convolutions. they can be tailored.  The layout of the weight tensor
is $COUT\times h \times w \times CIN$, which is similar to the caffe
layout \cite{Caffe} and different from \cite{tensorflow}. 
\begin{comment}
  (previously DPUV1 and DPUV3int8
  \cite{10.11451/3473334,abs-2110-04327}).
\end{comment}

The main differences from our previous compilers are the parameterized
representation of block sparsity and the capability to split tensors
and computations accordingly. All weights are statically prepared into
DDR and we move them explicitly towards the inner levels. Inputs and
outputs have designated space in DDR. DDR can and it will be used for
tensors spills.  The memory allocation to memtile is basically
coloring algorithms and some heuristics. In this architecture, we do
not allow {\em streaming} of neither data nor weights (because they
share space in memtile and input and output have different
consumption/production rates).

\subsection{AIE Hardware Abstraction}

\singlefigure{0.70}{AIE.png}{4x4 AIE representation}{fig:aie}

see Figure \ref{fig:aie}, we work with a mesh of 4x4 AIE2 cores
connected by 4 horizontal and 4 vertical interconnections.  We present
estimates for square 2x2, .. $i\times i$ .. 8x8 and rectangular shapes
are in the works. Each core has 8 banks memories for a total 64
KB. About six banks are used as input/output/weight buffers and two
banks are used as temporary space for kernels. Each core can request
and send data to its direct neighbors (if aware of connection and
control). Double buffering using ping/pong is used for inputs and
outputs.

There are four memtiles: each 512 KB and each is connected to one
columns and its direct neighbor column, or it is connected to a row
and its neighbor. The total amount of space is 2 MB. Memtile is a
circular buffer to exploit more flexible allocation. Note a $2 \times
2$ architecture will have one memtile per column and a total of two
memtiles (1 MB).

A Memtile can broadcast data per column or per row; it is a design
choice. We can dedicate one memtile for weights, one for activations,
or we can share it. In this work, we present results for shared
memtiles. To maximize the computation parallelism, every core will
write data per column into memtile.


\subsection{Subvolumes, Data Compression, and Data Movements}
The computation is split by memtile and thus by column (cores
columns). The output tensor is computed and split evenly by
width. Thus one memtile will store one partial tensor by width, each
core will compute different output channels, and the computation
streams the output tensor by rows and using ping/pong double
buffering. As often as possible, we store the weights in the core and
we reuse them (unless we need to {\em stream} the weight instead). The
cores set is a cohort and we always choose symmetric computations. We
do not merge two operations like convolution and max-pool and give
three columns to convolution and one column to max-pool. Other
solutions in the literature address this approach.

If we have the inputs, output, and weights in memtile, what is the
largest computation we can do in the AIE? The minimum computation is
one output channel and one row (i.e, by height). If this is not
possible, we try to reduce the size of the width (e.g., shaping the
tensor in memtile by using DDR spills) and we can manage to split the
input channels and to split the weights accordingly and prepare for
accumulattion. We call W-Split the distribution of tensor by columns
in the AIE mesh. We can COUT-split, this requires the partial transfer
of weights.  We can CIN-split when we need to split by input channel,
this is the last resort because it is also the most expensive
(accumulation of the outputs).

The subvolume describes the smallest shape of the weights we need to
manage and the largest computation in the core. We compress the weight
accordingly. Any data movement will always be a multiple of the
subvolume and is a single load. Such a compressed data will have the
same properties whether it is sparse or dense. The sparse data is
smaller, we compute fewer operations, we can compute larger
subproblems, and fewer data movements.


\subsection{Schedule and Memory Allocation}
During the scheduling of each layer, we evaluate what tensors can fit
in memtile. Here, activation and weight tensors share the space. It
means that an input tensor is distributed among the memtiles
identified by one starting address and a final address and so do the
weights. At each step, the memory allocation will check if we can
allocate (inputs, weights, and outputs). If we cannot, we evict all
tensors into DDR and then split/time the computation.

At the end of this stage, every tensor will have an address in memtile
or DDR (or both). If there are only DDR addresses, the compiler will
take the basic layer computation and, by heuristics, will split the
computation and the output tensor by width, output channel, height,
and input channel (no necessarily in this order). The heuristics have
a single objective to find the largest problem fitting the (each)
memory level. We deploy a recursive approach of tiling.  Formally,
$\dot{\sum}$ is a parallel loop and a W-split can be written as
follows:
\begin{equation}
  \Vc{Y} =  Conv(\Vc{X},\Vc{W}) = \dot{\sum}_w
  Conv(\Vc{X}_w,\Vc{W})
\end{equation}
The split is a function of the footprint. Before and after each
convolution, there will be an explicit data movement (optional). At
this stage each input, output, and weights have addresses associated
with each sub-computation. Then the code generation of each
$Conv(\Vc{X}_w,\Vc{W})$ is independent and recursive as needed. This
is a tree. If the convolution has strides or a large kernel, each
sub-convolution has overlap data; however, the sub-convolution has
defined addresses and data movements. For a W-split such as this, we
are computing the output by rows and the weights are reused (read
once). Note scheduling and memory allocation of graph layers is a hard
problem and often barely considered. We addressed it first (although
by heuristics).

\subsection{Code Generation }
The compiler creates a list of operations. These operations are
smaller and smaller and they can be executed from memtile to
memtile. There is a further decomposition using only AIE cores and it
is completely determined by the subvolume. Here, we show how we
generate code at this level and estimate time as in Figure
\ref{fig:singleconvestimate}.  This is the computation of a
convolution with top/bottom padding by height:
\begin{equation}
  \label{eq:convpadding}
  \Vc{Y}_{height} =   Conv(\Vc{X}_{h=0}) \dot{+} \dot{\sum}_{h=1}^9
  Conv(\Vc{X}_h) \dot{+} Conv(\Vc{X}_{h=10})
\end{equation}

An important feature of the current system is the concept of {\bf
  iteration} between memtile and core.  Using locks and chaining the
locks, we write a single instruction from the prospective of a single
core (as a SIMD instruction) and driving all cores at once for
multiple iterations $\dot{\sum}_{h=1}^i Conv(\Vc{X}_w)$ in Equation \ref{eq:convpadding}, the
ASM-like code follows:

{\small %footnotesize
\begin{verbatim}
  LOADFM Lock k_0 memtile addr core addr iter i
  CONV iteration      i
  WRITEFM Lock k_1 memtile addr core addr iter i
\end{verbatim}
} There is an implicit lock (say \verb2k_x2) that is used for the pong
and the system cycles between locks (\verb2k_x2 and \verb2k_02).
These three operations will be execute a number of iterations {\em i}
and, using a ping/pong, they will load different slices of data and
compute different slices of data.

Equation \ref{eq:convpadding} is encoded as follows: {\small
  %\footnotesize
\begin{verbatim}
  ## Head top pad < 50 us First comp block
  LOADFM Lock k_0 memtile addr_0 core addr iter 1
  CONV iteration 1
  WRITEFM Lock k_1 memtile addr_1 core addr iter 1
  ## Body iteration > 50 us < 150 us
  ## k_0 -> k_2 -> k_4 Lock Chain
  LOADFM Lock k_2 memtile addr_2 core addr iter 9
  CONV iteration 7
  WRITEFM Lock k_3 memtile addr_3 core addr iter 9
  ## tail bottom pad > 150 us Last computation block
  LOADFM Lock k_4 memtile addr_4 core addr iter 1
  CONV iteration 1
  WRITEFM Lock k_5 memtile addr_5 core addr iter 1
\end{verbatim}
 } We present in Figure \ref{fig:singleconvestimate} the execution
estimate of this code. At this stage, we have all the information. Per
layer, the code generation is a two pass process. First, we generate
code for the all loads/stores. Second we combine them into chains with
dependency, logically correct and as fast as possible.
\singlefigure{0.99}{singledenseconv.png}{Resnet single convolution
  with padding for 4x4: legend AIE: LOAD activation from DDR to
  memtile, LOADW weights from DDR to memtile, LOADFM activation from
  memtile to AIE2 cores, LOADWM weights from memtile to AIE2, WRITE
  from memtile to DDR, WRITEFM from AIE2 to memtile, COMP Computation.
}{fig:singleconvestimate}


\subsection{Time Estimation}
We explain how we capture the execution time and visualize it as in
Figure \ref{fig:singleconvestimate}. We start by the time estimates
for DDR to memtile communications. We have two communication types:
activations and weights. Per memtile there are two dedicated channels.
\begin{itemize}
 \item If we share activations and weights in the same memtile, we can
   use one channel for activations and one for weights. Thus the loads
   from DDR to memtile (LOAD and LOADW) are parallel with a bandwidth
   of 4 GBps. Writes from memtile to DDR (WRITE) can use both channels
   (8 GBps).

 \item If activations and weights go to different memtiles (for
   example weights to memtiles '0' and '3' and activations to '1' and
   '2'), each load is parallel and 8 GBps. Writes are identical.
\end{itemize}
Memtile are circular buffers so we can manage better the memory
allocation; however, we do not allow streaming.
   
The memtile connections with AIE cores are different. We assume a few
channels with again 4 GBps bandwidth. One memtile can broadcast
inputs to a cores column (and to the nearest neighbor). These
communications are for activations (LOADFM). One memtile can broadcast
to rows of cores (or the nearest neighbor), these are for weights
(LOADWM). We assume that the column and row communications are
parallel and each memtile core connection is parallel.

Every communication with iteration one is synchronous and sequential.
The load, convolution, and store is executed one after the other and
every core is independent.  For synchronization and for bookkeeping,
we assume that AIE2 weights communications (from memtiles) to cores are
synchronous and halting (LOADWM).

Every communication with iteration larger than one, we assume that
load (LOADFM), computation (COMP), and store (WRITEFM) are executed in
parallel and the overall execution time is the maximum of the
estimated time multiplied by the number of iterations.

We estimate the execution time of a subvolume (COMP) by the number of
operations divided by the maximum number of operations per cycle which
is in our scenario is $4\times 8 \times 8 = 256 $ operations per cycle
and 1 GHz frequency. This is very optimistic. The execution time is a
feature of the analysis. The estimate of the communications is more
compelling and we can easily mute the computation contribution. Note
however, that sparsity really reduces the computation time.

We do not account the wait and synchronization which are necessary to
reprogram the fabric. These are very expensive running on for a few
milliseconds.


\subsection{Convolution example}
\singlefigure{0.999}{singlesparseconv.png}{Resnet single convolution
  with padding and sparsity for 4x4 AIE }{fig:singleconvestimate2}

We present the time estimate for a convolution with padding, dense,
and with 50\% sparsity, see Figure \ref{fig:singleconvestimate} and
\ref{fig:singleconvestimate2}.  We explain in details how the time is
estimated and our assumptions about how the hardware works. First, we
load the weight and activations once in memtiles (LOAD and
LOADW). There are actually one load per memtile for a total of four
loads instructions per activation and weight. Because each load is to
a different memtile, they are parallel.  The activation and weight
communications are using two different channels and they are parallel
with 4 GBps bandwidth.  There is a single load of the weights from
memtiles to each core (LOADWM). This is done once and it is
blocking. We can relax this condition.



There are three computations (COMP); that is, top padding, the body,
and the bottom padding. For the top padding computation, there is the
sequential execution of loads to cores (LOADFM), computation (COMP),
and write to memtile (WRITEFM). This computation has iteration 1.

For the body computation, there are 9 iterations for three
instructions: we can see the load, the computation, and the write are
parallel. This is a simplification; there will be a little load poking
out at the beginning and a writing poking out at the end.  Then we
conclude with padding at the bottom of the computation.

For this convolution, the computation dominates the execution
time. Sparsity cuts the execution time by half: from 200 $\mu s$ to
130 $\mu s$. On one hand, there are convolutions that realize up to
$2\times$ performance; on the other, there are convolutions that are
dominated by the reading or writing. In the latter case, sparsity
helps only in space saving and may be spilling. In principle, now that
we know we can relax sparsity requirements for those convolutions (and
restart training).

%\SSinglefigure{2.0}{R4x4-4sharedsparse.png}{Resnet50 for 4x4 AIE with 50\%
%  sparse weights}{fig:estimate-sparse}

\section{Results}
\label{sec:experiments}
\begin{table}[htb]
  \caption{Execution Time estimates}
  \label{tab_perf}
\begin{center} 
\begin{tabular}{|l|l|l|l|l|}
  \hline
  AIE2 & Model  & Dense sec      & Sparse sec      \\ \hline\hline
  2x2   & Resnet & 2.492347e-02  & 1.582626e-02 \\ \hline
  3x3   &  & 1.269543e-02  & 8.661490e-03 \\ \hline
  4x4   &  &  1.077318e-02 & 7.064918e-03 \\ \hline
  5x5   &  &  failed       & 4.303485e-03 \\ \hline
  6x6   &  &  5.712521e-03 & 4.490127e-03 \\ \hline
  7x7   &  &  4.205991e-03 & 3.212234e-03 \\ \hline
  8x8   &  &  6.376768e-03 & 4.602027e-03 \\ \hline \hline
  2x2   & IncV3  & 4.283837e-02  & 2.440544e-02 \\ \hline
  3x3   &   & 2.386600e-02  & 1.422390e-02 \\ \hline
  4x4   &   &  1.740967e-02 & 1.012540e-02 \\ \hline
  5x5   &   &  9.690552e-03 & failed       \\ \hline
  6x6   &   &  1.063962e-02 & 6.439692e-03 \\ \hline
  7x7   &   &  8.727651e-03 & failed       \\ \hline
  8x8   &   &  9.093276e-03 & 5.666152e-03 \\ \hline \hline
  2x2   & VGG16  & 4.476212e-02  & 2.608593e-02 \\ \hline
  3x3   &   & failed        & 1.002015e-02 \\ \hline
  4x4   &   &  1.371000e-02 & 8.852128e-03 \\ \hline
  5x5   &   &  failed       & 4.336479e-03 \\ \hline
  6x6   &   &  failed       & 5.770197e-03 \\ \hline
  7x7   &   &  7.455440e-03 & 5.288551e-03 \\ \hline
  8x8   &   &  9.203393e-03 & 6.502333e-03 \\ \hline \hline
          
\end{tabular}
\end{center}
\end{table}

In Table \ref{tab_perf}, we present the performance of sparsity
applied to all the convolutions (except the first one) for Resnet 50,
Inception V3, and VGG16.

%\DDoublefigure{0.80}{Iv36x6-6shareddense.png}{Iv36x6-6sharedsparse.png}{Inception V3 for 6x6 AIE
%  with dense (left/top)  weights and sparse (right/bottom) }{fig:incv3-estimate-dense}

When we generate the code, we compute the execution time. When we
inspect the assembly code, we will find time information in the
context whether or not each instruction contribute directly. For data
movement to and from DDR and memtile, we reduce the contribution (sum
directly).

For memtile to and from core communications and core computations, we
create a time series. All of this will be an attribute of the layer
(node in the graph computation).  To create a complete time estimate,
we take the schedule of the computation of the graph, visit each node
accordingly to the schedule, write a {\em json} file describing the
time series.  By {\em javascript}, we visualize the time series using
a browser. The figures in this paper are generated directly.

For a $4\times 4$ AIE set up, Resnet 50 fits in memtile from the
beginning to the last operation (beside the first convolution and this
is by costruction). The estimates advantage by sparsity is almost
completely achieved.

Corner cases are represented as failure in Table \ref{tab_perf}. Some
cases is because of inability to break the weight tensor evenly.
Sometime is for incorrect data management especially for prime
numbers. These are all issues will be address as the technology
matures. Please, note that VGG16 using 8x8 is slower than 7x7 by using
sparse.  It may happen because a symmetric computation my use too
small sub-volume computations and thus more iterations for the same
amount of data transfers. We will provide a full discussion if space
permits.

\section{Conclusions and Context}
This is a multifaceted problem and we present a complete solution from
training techniques, compilers, code generation, HW definition, and
time estimations. It is a vertical software system, more complex than
just a prototype, and it is used for the validation and comparison of
different HW designs.

This could be seen as a quantization and sparsification problem. For
example, how we can reduce the footprint of a CNN network. There are
post training techniques that are targeting quantization and
unstructured sparsity \cite{frantar2023gptq} and all the references
within. We need to be more aggressive and training for it (as starting
point \cite{abs-2102-11289}).  Our sparsity is not really a property
of the model, software can describe it, and the hardware can take
advantage; however, we do not need specific hardware support at
instruction level.

This work stemmed from a collaboration between Xilinx and Numenta and
the idea set forward by the authors in \cite{ahmad2019dense}, where
the models are too dense and there is a lot to improve.


The difficulty of generate code for complex architectures can be
described and solved in different ways. There are recent attempts of
introducing SW/HW solution for spatial processors like ours
\cite{Huang2021CoSASB,Russo2023MemoryAwareDA,Cai2023InterlayerSS}.
Usually major attention is given only to matrix multiplication and
GPUs \cite{Gray2017GPUKF} \cite{li2023popsparse}, we can work on only static
sparsity at this time.








%%
%% The next two lines define the bibliography style to be used, and
%% the bibliography file.
\bibliographystyle{ACM-Reference-Format}
\bibliography{ref}



\end{document}
\endinput
%%
%% End of file `sample-sigconf.tex'.
